

\section*{P\+WP }

\subsection*{Introduction }

The procotol works by exchanging messages with other peer and by mantaining a {\itshape state} describing the peer connection state.

For each peer connection, the client should know -\/am\+\_\+chocking \+: If the client is chocking the peer -\/am\+\_\+interested \+: If the client is interested in the peer -\/peer\+\_\+choking \+: If the peer is chocking the client -\/peer\+\_\+interested \+: If the peer is interested

During the protocol execution, all integers in the peer wire protocol are encoded as four byte big-\/endian values.

\subsection*{Handshaking }

After enstablishing a connection, the first thing to do according to the protocol is the handshake procedure. The handshake is crafted referring to this table

\tabulinesep=1mm
\begin{longtabu} spread 0pt [c]{*{3}{|X[-1]}|}
\hline
\rowcolor{\tableheadbgcolor}\textbf{ Name }&\textbf{ Size }&\textbf{ Description  }\\\cline{1-3}
\endfirsthead
\hline
\endfoot
\hline
\rowcolor{\tableheadbgcolor}\textbf{ Name }&\textbf{ Size }&\textbf{ Description  }\\\cline{1-3}
\endhead
pstrlen &1 &string length of $<$pstr$>$, as a single raw byte (19) \\\cline{1-3}
pstr &19 &the string \char`\"{}\+Bit\+Torrent protocol\char`\"{} \\\cline{1-3}
reserved &8 &eight (8) reserved bytes. \\\cline{1-3}
info\+\_\+hash &20 &20-\/byte S\+H\+A1 hash of the info key in the metainfo file. \\\cline{1-3}
peer\+\_\+id &20 &Peer-\/\+ID string \\\cline{1-3}
\end{longtabu}
The reserved bytes is tipically used to communicate that the client is capable of managing a non-\/standard protocol by setting a specific bit to 1 alpha\+Torrent simply print a warning in case of a non-\/standard client is communicating.

So after sending the handshake request, the client listen for a response, and when arrives, it start the \hyperlink{namespacepwp_a58c780495f2139a56b95662dc7c0345f}{pwp\+::verify\+\_\+handshake} routing, which basically check if all element of the table is valid (according to the protocol).

\begin{DoxyNote}{Note}
Because peer-\/\+ID matching often fail during live test, in case of a non matching id nothing is done (excepting print a warning).
\end{DoxyNote}
\subsection*{Pre-\/\+Loop phase }

After the handshake procedure is completed, the peer is flagged as \char`\"{}active\+\_\+peer\char`\"{} and some messages are exchanged before starting donwload/upload pieces.

First of all, the client should check if a bitfiled was sended by the peer. This part is optional so it must be properly handled. Then an \char`\"{}interested\char`\"{} and \char`\"{}unchocked\char`\"{} message are sended in order to enable piece downloading from the peer. A timer which call the \hyperlink{namespacepwp__msg_a9a577f5a53b823d83bb4694f1ebf141e}{pwp\+\_\+msg\+::send\+\_\+keep\+\_\+alive} is started in order to avoid a connection closing.

\subsection*{P\+WP Loop }

The P\+WP loop consist in two parts \+:
\begin{DoxyItemize}
\item Async Receive
\item Sync Send
\end{DoxyItemize}

The {\bfseries async handler}(pwp\+::read\+\_\+msg\+\_\+handler) is called when 4 bytes are received. After checking that the msg\+\_\+len is greater than zero, a new byte (the msg\+\_\+id\mbox{[}\hyperlink{namespacepwp__msg_a0b9a29508f00a30e5138d2b78f4b1daf}{pwp\+\_\+msg\+::msg\+\_\+id}\mbox{]}) is readed. The message is then switched based on its id. If a message like \char`\"{}(not) interested\char`\"{} and \char`\"{}(un)chocked\char`\"{} is received then the conresponding structure inside the \hyperlink{structpwp_1_1peer__connection}{pwp\+::peer\+\_\+connection} is updated (client\+\_\+state and peer\+\_\+state). \begin{DoxyNote}{Note}
R\+E\+Q\+U\+E\+ST, P\+O\+RT and C\+A\+N\+C\+EL message are ignored since seeding is not supported. When a P\+I\+E\+CE message arrives the response is passeto to the \hyperlink{namespacefileio_acc37418d350d6d36c3f691c99de9cf39}{fileio\+::save\+\_\+block} function wich is responsible of putting the piece in the right position inside the file. After finishing the procedure, another async read is started in order to continually parse received data.
\end{DoxyNote}
The {\bfseries Sync Sender} insted is placed inside a loop that exit only if the send fails (dead peer). The \hyperlink{namespacepwp__msg_ab578b213d293636d33efc24382f16b25}{pwp\+\_\+msg\+::sender} simply compare the bitfiled of the peer with the client ones and send a R\+E\+Q\+U\+E\+ST message to the peer referring to the client\textquotesingle{}s missing pieces. 